% !TEX encoding = UTF-8 Unicode
\documentclass[a4paper]{article}

\usepackage[utf8]{inputenc}
\usepackage{float}
\usepackage[slovene,english]{babel}
\usepackage{erk}
\usepackage{times}
\usepackage{graphicx}
\usepackage{amsmath}
\usepackage[top=22.5mm, bottom=22.5mm, left=22.5mm, right=22.5mm]{geometry}
\usepackage{hyperref}

\usepackage{xcolor}

\pagecolor[rgb]{0.1,0.1,0.1} %black
\color[rgb]{0.9,0.9,0.95} %grey

% lokalne definicije
\def\footnotemark{} % da se izognemo opombi na naslovnici (erk.sty trik)

\title{Poročilo druge laboratorijske vaje pri predmetu Informacija in kodi}

\author{Rok Prezelj
}

\affiliation{Univerza v Ljubljani, Fakulteta za elektrotehniko}

\email{E-pošta: rp0067@student.uni-lj.si}

\begin{document}

\maketitle

\begin{abstract}
\noindent\textbf{Povzetek.}
Pri laboratorijski vaji smo analizirali kodne tabele, ki podpirajo slovenske šumnike (IBM-852, ISO-8859-2, Windows-1250, MacCE) ter Unicode kodiranja UTF-8, UTF-16LE in UTF-16BE. V drugem delu smo implementirali program, ki iz zaporedja Unicode kodnih točk ročno ustvari pravilno UTF-8 kodirano besedilo.
\end{abstract}

\selectlanguage{slovene}

\section{Uvod}
Kodiranje znakov omogoča shranjevanje in prenos besedilnih podatkov v digitalni obliki. V preteklosti so različna okolja uporabljala lastne 8-bitne kodne tabele (npr. IBM-852, ISO-8859-2, Windows-1250, MacCE), ki razširijo 7-bitni ASCII z nacionalnimi znaki. Ker isti bajt v različnih tabelah pogosto predstavlja različne znake, to povzroča težave pri izmenjavi podatkov.

Standard Unicode vsakemu znaku dodeli enolično kodno točko, kodirni sistemi UTF-8, UTF-16LE in UTF-16BE pa določajo pretvorbo v zaporedje bajtov~\cite{unicode}. Pri vaji smo analizirali kodiranje slovenskih šumnikov (Č, Š, Ž, č, š, ž) v izbranih kodnih tabelah, nato pa implementirali ročno pretvorbo iz Unicode kodnih točk v UTF-8.

\section{Kodni standardi}

V prvem delu vaje smo obravnavali naslednje kodne tabele, ki podpirajo slovenske šumnike:

\begin{itemize}
    \item \textbf{IBM-852}: znan tudi kot DOS Central European, je standard za kodiranje znakov, ki ga je razvil IBM. Razvit je bil predvsem za srednjeevropske jezike, ki uporabljajo latinico~\cite{IBM-852}.

    \item \textbf{ISO-8859-2}: del standarda ISO/IEC 8859, namenjen jezikom srednje Evrope. V zgornji polovici kode vsebuje znake, potrebne za slovenščino, hrvaščino, češčino in slovaščino~\cite{ISO-8859-2}.

    \item \textbf{Windows-1250}: je kodna tabela, ki se uporablja v operacijskem sistemu Microsoft Windows za prikazovanje besedil v srednjeevropskih in vzhodnoevropskih jezikih, ki uporabljajo latinico~\cite{WIN-1250}.

    \item \textbf{MacCE}: kodiranje Macintosh Central European se uporablja v računalnikih Apple Macintosh za prikazovanje besedil v srednjeevropskih in jugovzhodnoevropskih jezikih, ki uporabljajo latinico~\cite{MacCE}.

    \item \textbf{UTF-8}: kodiranje Unicode kodnih točk s spremenljivo dolžino. ASCII znaki so kodirani z enim bajtom, ostali z 2, 3 ali 4 bajti~\cite{rfc3629}.

    \item \textbf{UTF-16LE/UTF-16BE}: 16-bitni kodiranji Unicode, ki za znake v osnovni večjezični ravnini uporabita eno 16-bitno enoto. Razlikujeta se po vrstnem redu bajtov (little-endian in big-endian)~\cite{unicode1}.
\end{itemize}

\section{Metodologija}
Pri izvedbi laboratorijske vaje smo uporabili programski jezik Python zaradi preproste uporabe.

\subsection{Kodne tabele slovenskih znakov}
Za 8-bitne kodne tabele smo uporabili Pythonovo funkcijo \texttt{encode()} na nizu \texttt{"ČŠŽčšž"} in iz dobljenih bajtov izračunali desetiške, šestnajstiške in dvojiške predstavitve.

\subsection{Program za pretvorbo kodnih točk v UTF-8}
V drugem delu vaje smo implementirali program v Pythonu, ki iz vhodne datoteke prebere zaporedje Unicode kodne točke, ločene z vejicami, in jih pretvori v UTF-8 besedilo. Program deluje po naslednjih korakih:

\begin{enumerate}
    \item Prebere vsebino datoteke kot ASCII besedilo in jo pretvori v list z števili.
    \item Za vsako število preveri, ali gre za veljavno Unicode kodno točko [$\texttt{0}, \texttt{0010FFFF}$], ter izključeno surrogate območje [$\texttt{D800}, \texttt{DFFF}$].
    \item Na podlagi vrednosti kodne točke ročno izračuna UTF-8 predstavitev:
    \begin{itemize}
        \item 1 bajt: \texttt{0xxxxxxx},
        \item 2 bajta: \texttt{110xxxxx 10xxxxxx},
        \item 3 bajti: \texttt{1110xxxx 10xxxxxx 10xxxxxx},
        \item 4 bajti: \texttt{11110xxx 10xxxxxx 10xxxxxx 10xxxxxx}.
    \end{itemize}
    \item UTF-8 bajte zapiše v izhodno datoteko v binarnem načinu (\texttt{"wb"}).
\end{enumerate}

\section{Rezultati}
V nadaljevanju so prikazane kodne zamenjave šumnikov ter rezultat pretvorbe Unicode kodnih točk v UTF-8.

\subsection{8-bitne kodne tabele}
V preglednicah~\ref{tab:ibm852}–\ref{tab:macce} so podane kodne zamenjave za slovenske šumnike v izbranih 8-bitnih kodnih tabelah. Vrednosti smo generirali z \texttt{.encode(\dots)}.

\begin{table}[H]
\centering
\begin{tabular}{|c|c|c|c|}
\hline
znak & desetiško & šestnajstiško & binarno \\ \hline
č & 159 & 9F & 10011111 \\
š & 231 & E7 & 11100111 \\
ž & 167 & A7 & 10100111 \\
Č & 172 & AC & 10101100 \\
Š & 230 & E6 & 11100110 \\
Ž & 166 & A6 & 10100110 \\ \hline
\end{tabular}
\caption{Kodne zamenjave šumnikov po tabeli IBM-852.}
\label{tab:ibm852}
\end{table}

% ISO-8859-2
\begin{table}[H]
\centering
\begin{tabular}{|c|c|c|c|}
\hline
znak & desetiško & šestnajstiško & binarno \\ \hline
č & 232 & E8 & 11101000 \\
š & 185 & B9 & 10111001 \\
ž & 190 & BE & 10111110 \\
Č & 200 & C8 & 11001000 \\
Š & 169 & A9 & 10101001 \\
Ž & 174 & AE & 10101110 \\ \hline
\end{tabular}
\caption{Kodne zamenjave šumnikov po tabeli ISO-8859-2.}
\label{tab:latin2}
\end{table}

% Windows-1250
\begin{table}[H]
\centering
\begin{tabular}{|c|c|c|c|}
\hline
znak & desetiško & šestnajstiško & binarno \\ \hline
č & 232 & E8 & 11101000 \\
š & 154 & 9A & 10011010 \\
ž & 158 & 9E & 10011110 \\
Č & 200 & C8 & 11001000 \\
Š & 138 & 8A & 10001010 \\
Ž & 142 & 8E & 10001110 \\ \hline
\end{tabular}
\caption{Kodne zamenjave šumnikov po tabeli Windows-1250.}
\label{tab:win1250}
\end{table}

% MacCE
\begin{table}[H]
\centering
\begin{tabular}{|c|c|c|c|}
\hline
znak & desetiško & šestnajstiško & binarno \\ \hline
č & 139 & 8B & 10001011 \\
š & 228 & E4 & 11100100 \\
ž & 236 & EC & 11101100 \\
Č & 137 & 89 & 10001001 \\
Š & 225 & E1 & 11100001 \\
Ž & 235 & EB & 11101011 \\ \hline
\end{tabular}
\caption{Kodne zamenjave šumnikov po tabeli MacCE.}
\label{tab:macce}
\end{table}

% UTF-8
\begin{table}[H]
\centering
\begin{tabular}{|c|c|c|c|}
\hline
znak & deset. & šest. & binarno \\ \hline
č & 50317 & C4 8D & 11000100 10001101 \\
š & 50593 & C5 A1 & 11000101 10100001 \\
ž & 50622 & C5 BE & 11000101 10111110 \\
Č & 50316 & C4 8C & 11000100 10001100 \\
Š & 50592 & C5 A0 & 11000101 10100000 \\
Ž & 50621 & C5 BD & 11000101 10111101 \\ \hline
\end{tabular}
\caption{UTF-8 kodne zamenjave za slovenske šumnike (bajti interpretirani kot celo število).}
\label{tab:utf8}
\end{table}

V UTF-8 se slovenski šumniki kodirajo z dvema bajtoma. V preglednici~\ref{tab:utf8} je za vsak znak podana njegova Unicode vrednost in pripadajoča UTF-8 predstavitev.

% UTF-16LE
\begin{table}[H]
\centering
\begin{tabular}{|c|c|c|c|}
\hline
znak & deset. & šest. & binarno \\ \hline
č & 3329 & 0D 01 & 00001101 00000001 \\
š & 24833 & 61 01 & 01100001 00000001 \\
ž & 32257 & 7E 01 & 01111110 00000001 \\
Č & 3073 & 0C 01 & 00001100 00000001 \\
Š & 24577 & 60 01 & 01100000 00000001 \\
Ž & 32001 & 7D 01 & 01111101 00000001 \\ \hline
\end{tabular}
\caption{Kodne zamenjave šumnikov v kodiranju UTF-16LE (bajta interpretirana kot malo-endiško celo število).}
\label{tab:utf16le}
\end{table}

% UTF-16BE
\begin{table}[H]
\centering
\begin{tabular}{|c|c|c|c|}
\hline
znak & deset. & šest. & binarno \\ \hline
č & 269 & 01 0D & 00000001 00001101 \\
š & 353 & 01 61 & 00000001 01100001 \\
ž & 382 & 01 7E & 00000001 01111110 \\
Č & 268 & 01 0C & 00000001 00001100 \\
Š & 352 & 01 60 & 00000001 01100000 \\
Ž & 381 & 01 7D & 00000001 01111101 \\ \hline
\end{tabular}
\caption{Kodne zamenjave šumnikov v kodiranju UTF-16BE (bajta interpretirana kot veliko-endiško celo število).}
\label{tab:utf16be}
\end{table}

V UTF-16 se znaki iz osnovne večjezične ravnine predstavijo z eno 16-bitno enoto, enako Unicode vrednosti. V tabelah~\ref{tab:utf16le} in~\ref{tab:utf16be} je prikazana 16-bitna binarna predstavitev kodnih točk šumnikov; pri UTF-16LE in UTF-16BE se bajta te 16-bitne vrednosti zamenjata.

Iz preglednic je razvidno, da se šumniki v različnih 8-bitnih kodnih tabelah nahajajo na različnih mestih, zato ista bajtna vrednost ne predstavlja nujno istega znaka.

\subsection{Pretvorba datoteke v UTF-8}
Vhodna datoteka vsebuje zaporedje Unicode kodnih točk, ki predstavljajo prvi stavek wikipedije o UTF-8 v več jezikih. Program je datoteko prebral kot ASCII besedilo ter ročno pretvoril v UTF-8 in rezultat zapisal v izhodno datoteko. Generirano datoteko smo uspešno odprli z urejevalnikom besedila, pri čemer so se vsi znaki prikazali pravilno.

\section{Zaključek}
V vaji smo primerjali kodne tabele za slovenske šumnike ter uspešno implementirali ročno pretvorbo Unicode kodnih točk v UTF-8, kar je potrdilo enoličnost in zanesljivost Unicode kodiranja.


\small
\begin{thebibliography}{1}

\bibitem{ucbenik}
N. Pavešić, \emph{Informacija in kodi}, 2. spremenjena in dopolnjena izdaja.
Ljubljana: Založba Fakultete za elektrotehniko in Fakultete za računalništvo in informatiko, 2010.
ISBN 978-961-243-145-7.

\bibitem{IBM-852}
Localizely, Character Encoding: CP 852
[Na spletu]. Dostopno na: \url{https://localizely.com/character-encodings/cp852/}
[Dostopano: 22. november 2025]

\bibitem{ISO-8859-2}
\emph{ISO/IEC 8859-2:1999} Information technology — 8-bit single-byte coded graphic character sets — Part 2: Latin alphabet No. 2
[Na spletu]. Dostopno na: \url{https://www.iso.org/obp/ui/en/#iso:std:iso-iec:8859:-2:ed-1:v1:en}
[Dostopano: 22. november 2025]

\bibitem{WIN-1250}
\emph{Wikipedia} Windows-1250
[Na spletu]. Dostopno na: \url{https://en.wikipedia.org/wiki/Windows-1250}
[Dostopano: 22. november 2025]

\bibitem{MacCE}
\emph{Everything explained today} Mac OS Central European encoding explained
[Na spletu]. Dostopno na: \url{https://everything.explained.today/Mac_OS_Central_European_encoding/}
[Dostopano: 22. november 2025]

\bibitem{unicode1}
The Unicode Consortium,
\emph{The Unicode Standard, Version 15.0.0}.
Mountain View, CA: The Unicode Consortium, 2022.
ISBN 978-1-936213-32-0.

\bibitem{unicode}
The Unicode Consortium, \emph{The Unicode Standard}, v. 15.0.
[Na spletu]. Dostopno na: \url{https://www.unicode.org/standard/standard.html}

\bibitem{rfc3629}
F. Yergeau, ``UTF-8, a transformation format of ISO 10646,''
\emph{IETF RFC 3629}, nov. 2003.
[Na spletu]. Dostopno na: \url{https://www.rfc-editor.org/rfc/rfc3629}

\bibitem{python}
Python Software Foundation, \emph{Python 3 Documentation}.
[Na spletu]. Dostopno na: \url{https://docs.python.org/3/}

\end{thebibliography}

\end{document}
