% !TEX encoding = UTF-8 Unicode
\documentclass[a4paper]{article}

\usepackage[utf8]{inputenc}
\usepackage{float}
\usepackage[slovene,english]{babel}
\usepackage{erk}
\usepackage{times}
\usepackage{graphicx}
\usepackage{amsmath}
\usepackage[top=22.5mm, bottom=22.5mm, left=22.5mm, right=22.5mm]{geometry}
\usepackage{hyperref}


\usepackage{fontspec} % enables unicode + system fonts
% Chinese (Simplified + Traditional)
\newfontfamily\cjkfont{Noto Serif CJK SC}
% Korean
\newfontfamily\krfont{Noto Serif CJK KR}
% Greek (if you want explicit control)
\newfontfamily\grfont{Noto Serif}


\usepackage{xcolor}

% \pagecolor[rgb]{0.1,0.1,0.1} %black
% \color[rgb]{0.9,0.9,0.95} %grey

% lokalne definicije
\def\footnotemark{} % da se izognemo opombi na naslovnici (erk.sty trik)

\title{Poročilo druge laboratorijske vaje pri predmetu Informacija in kodi}

\author{Rok Prezelj
}

\affiliation{Univerza v Ljubljani, Fakulteta za elektrotehniko}

\email{E-pošta: rp0067@student.uni-lj.si}

\begin{document}

\maketitle

\begin{abstract}
\noindent\textbf{Povzetek.}
Pri laboratorijski vaji smo analizirali kodne tabele, ki podpirajo slovenske šumnike (IBM-852, ISO-8859-2, Windows-1250, MacCE) ter Unicode kodiranja UTF-8, UTF-16LE in UTF-16BE. V drugem delu smo implementirali program, ki iz zaporedja Unicode kodnih točk ročno ustvari pravilno UTF-8 kodirano besedilo.
\end{abstract}

\selectlanguage{slovene}

\section{Uvod}
Kodiranje znakov omogoča shranjevanje in prenos besedilnih podatkov v digitalni obliki. V preteklosti so različna okolja uporabljala lastne 8-bitne kodne tabele (npr. IBM-852, ISO-8859-2, Windows-1250, MacCE), ki razširijo 7-bitni ASCII z nacionalnimi znaki. Ker isti bajt v različnih tabelah pogosto predstavlja različne znake, to povzroča težave pri izmenjavi podatkov.

Standard Unicode vsakemu znaku dodeli enolično kodno točko, kodirni sistemi UTF-8, UTF-16LE in UTF-16BE pa določajo pretvorbo v zaporedje bajtov~\cite{unicode}. Pri vaji smo analizirali kodiranje slovenskih šumnikov (Č, Š, Ž, č, š, ž) v izbranih kodnih tabelah, nato pa implementirali ročno pretvorbo iz Unicode kodnih točk v UTF-8.

\section{Kodni standardi}

V prvem delu vaje smo obravnavali naslednje kodne tabele, ki podpirajo slovenske šumnike:

\begin{itemize}
    \item \textbf{IBM-852}: znan tudi kot DOS Central European, je standard za kodiranje znakov, ki ga je razvil IBM. Razvit je bil predvsem za srednjeevropske jezike, ki uporabljajo latinico~\cite{IBM-852}.

    \item \textbf{ISO-8859-2}: del standarda ISO/IEC 8859, namenjen jezikom srednje Evrope. V zgornji polovici kode vsebuje znake, potrebne za slovenščino, hrvaščino, češčino in slovaščino~\cite{ISO-8859-2}.

    \item \textbf{Windows-1250}: je kodna tabela, ki se uporablja v operacijskem sistemu Microsoft Windows za prikazovanje besedil v srednjeevropskih in vzhodnoevropskih jezikih, ki uporabljajo latinico~\cite{WIN-1250}.

    \item \textbf{MacCE}: kodiranje Macintosh Central European se uporablja v računalnikih Apple Macintosh za prikazovanje besedil v srednjeevropskih in jugovzhodnoevropskih jezikih, ki uporabljajo latinico~\cite{MacCE}.

    \item \textbf{UTF-8}: kodiranje Unicode kodnih točk s spremenljivo dolžino. ASCII znaki so kodirani z enim bajtom, ostali z 2, 3 ali 4 bajti~\cite{rfc3629}.

    \item \textbf{UTF-16LE/UTF-16BE}: 16-bitni kodiranji Unicode, ki za znake v osnovni večjezični ravnini uporabita eno 16-bitno enoto. Razlikujeta se po vrstnem redu bajtov (little-endian in big-endian)~\cite{unicode1}.
\end{itemize}

\section{Metodologija}
Pri izvedbi laboratorijske vaje smo uporabili programski jezik Python zaradi preproste uporabe.

\subsection{Kodne tabele slovenskih znakov}
Za 8-bitne kodne tabele smo uporabili Pythonovo funkcijo \texttt{encode()} na nizu \texttt{"ČŠŽčšž"} in iz dobljenih bajtov izračunali desetiške, šestnajstiške in dvojiške predstavitve.

\subsection{Program za pretvorbo kodnih točk v UTF-8}
V drugem delu vaje smo implementirali program v Pythonu, ki iz vhodne datoteke prebere zaporedje Unicode kodne točke, ločene z vejicami, in jih pretvori v UTF-8 besedilo. Program deluje po naslednjih korakih:

\begin{enumerate}
    \item Prebere vsebino datoteke kot ASCII besedilo in jo pretvori v list z števili.
    \item Za vsako število preveri, ali gre za veljavno Unicode kodno točko [$\texttt{0}, \texttt{0010FFFF}$], ter izključeno surrogate območje [$\texttt{D800}, \texttt{DFFF}$].
    \item Na podlagi vrednosti kodne točke ročno izračuna UTF-8 predstavitev:
    \begin{itemize}
        \item 1 bajt: \texttt{0xxxxxxx},
        \item 2 bajta: \texttt{110xxxxx 10xxxxxx},
        \item 3 bajti: \texttt{1110xxxx 10xxxxxx 10xxxxxx},
        \item 4 bajti: \texttt{11110xxx 10xxxxxx 10xxxxxx 10xxxxxx}.
    \end{itemize}
    \item UTF-8 bajte zapiše v izhodno datoteko v binarnem načinu (\texttt{"wb"}).
\end{enumerate}

\section{Rezultati}
V nadaljevanju so prikazane kodne zamenjave šumnikov ter rezultat pretvorbe Unicode kodnih točk v UTF-8.

\subsection{8-bitne kodne tabele}
V preglednicah~\ref{tab:ibm852}–\ref{tab:macce} so podane kodne zamenjave za slovenske šumnike v izbranih 8-bitnih kodnih tabelah. Vrednosti smo generirali z \texttt{.encode(\dots)}.

\begin{table}[H]
\centering
\begin{tabular}{|c|c|c|c|}
\hline
znak & desetiško & šestnajstiško & binarno \\ \hline
č & 159 & 9F & 10011111 \\
š & 231 & E7 & 11100111 \\
ž & 167 & A7 & 10100111 \\
Č & 172 & AC & 10101100 \\
Š & 230 & E6 & 11100110 \\
Ž & 166 & A6 & 10100110 \\ \hline
\end{tabular}
\caption{Kodne zamenjave šumnikov po tabeli IBM-852.}
\label{tab:ibm852}
\end{table}

% ISO-8859-2
\begin{table}[H]
\centering
\begin{tabular}{|c|c|c|c|}
\hline
znak & desetiško & šestnajstiško & binarno \\ \hline
č & 232 & E8 & 11101000 \\
š & 185 & B9 & 10111001 \\
ž & 190 & BE & 10111110 \\
Č & 200 & C8 & 11001000 \\
Š & 169 & A9 & 10101001 \\
Ž & 174 & AE & 10101110 \\ \hline
\end{tabular}
\caption{Kodne zamenjave šumnikov po tabeli ISO-8859-2.}
\label{tab:latin2}
\end{table}

% Windows-1250
\begin{table}[H]
\centering
\begin{tabular}{|c|c|c|c|}
\hline
znak & desetiško & šestnajstiško & binarno \\ \hline
č & 232 & E8 & 11101000 \\
š & 154 & 9A & 10011010 \\
ž & 158 & 9E & 10011110 \\
Č & 200 & C8 & 11001000 \\
Š & 138 & 8A & 10001010 \\
Ž & 142 & 8E & 10001110 \\ \hline
\end{tabular}
\caption{Kodne zamenjave šumnikov po tabeli Windows-1250.}
\label{tab:win1250}
\end{table}

% MacCE
\begin{table}[H]
\centering
\begin{tabular}{|c|c|c|c|}
\hline
znak & desetiško & šestnajstiško & binarno \\ \hline
č & 139 & 8B & 10001011 \\
š & 228 & E4 & 11100100 \\
ž & 236 & EC & 11101100 \\
Č & 137 & 89 & 10001001 \\
Š & 225 & E1 & 11100001 \\
Ž & 235 & EB & 11101011 \\ \hline
\end{tabular}
\caption{Kodne zamenjave šumnikov po tabeli MacCE.}
\label{tab:macce}
\end{table}

% UTF-8
\begin{table}[H]
\centering
\begin{tabular}{|c|c|c|c|}
\hline
znak & deset. & šest. & binarno \\ \hline
č & 50317 & C4 8D & 11000100 10001101 \\
š & 50593 & C5 A1 & 11000101 10100001 \\
ž & 50622 & C5 BE & 11000101 10111110 \\
Č & 50316 & C4 8C & 11000100 10001100 \\
Š & 50592 & C5 A0 & 11000101 10100000 \\
Ž & 50621 & C5 BD & 11000101 10111101 \\ \hline
\end{tabular}
\caption{UTF-8 kodne zamenjave za slovenske šumnike (bajti interpretirani kot celo število).}
\label{tab:utf8}
\end{table}

V UTF-8 se slovenski šumniki kodirajo z dvema bajtoma. V preglednici~\ref{tab:utf8} je za vsak znak podana njegova Unicode vrednost in pripadajoča UTF-8 predstavitev.

% UTF-16LE
\begin{table}[H]
\centering
\begin{tabular}{|c|c|c|c|}
\hline
znak & deset. & šest. & binarno \\ \hline
č & 3329 & 0D 01 & 00001101 00000001 \\
š & 24833 & 61 01 & 01100001 00000001 \\
ž & 32257 & 7E 01 & 01111110 00000001 \\
Č & 3073 & 0C 01 & 00001100 00000001 \\
Š & 24577 & 60 01 & 01100000 00000001 \\
Ž & 32001 & 7D 01 & 01111101 00000001 \\ \hline
\end{tabular}
\caption{Kodne zamenjave šumnikov v kodiranju UTF-16LE (bajta interpretirana kot malo-endiško celo število).}
\label{tab:utf16le}
\end{table}

% UTF-16BE
\begin{table}[H]
\centering
\begin{tabular}{|c|c|c|c|}
\hline
znak & deset. & šest. & binarno \\ \hline
č & 269 & 01 0D & 00000001 00001101 \\
š & 353 & 01 61 & 00000001 01100001 \\
ž & 382 & 01 7E & 00000001 01111110 \\
Č & 268 & 01 0C & 00000001 00001100 \\
Š & 352 & 01 60 & 00000001 01100000 \\
Ž & 381 & 01 7D & 00000001 01111101 \\ \hline
\end{tabular}
\caption{Kodne zamenjave šumnikov v kodiranju UTF-16BE (bajta interpretirana kot veliko-endiško celo število).}
\label{tab:utf16be}
\end{table}

V tabelah~\ref{tab:utf16le} in~\ref{tab:utf16be} je prikazana 16-bitna binarna predstavitev kodnih točk , pri UTF-16LE in UTF-16BE se bajta te 16-bitne vrednosti zamenjata.

Iz preglednic je razvidno, da se šumniki v različnih kodnih tabelah nahajajo na različnih mestih, zato ista bajtna vrednost ne predstavlja nujno istega znaka.

\subsection{Pretvorba datoteke v UTF-8}
Vhodna datoteka vsebuje zaporedje Unicode kodnih točk, ki predstavljajo prvi stavek wikipedije o UTF-8 v več jezikih. Program je datoteko prebral kot ASCII besedilo ter ročno pretvoril v UTF-8 in rezultat zapisal v izhodno datoteko. Rezultati prvega dela naloge so vidni v prilogi \ref{nal2.1} ter v tabelah \ref{tab:nal2:eu}-\ref{tab:nal2:korean}.

\section{Zaključek}
V vaji smo primerjali kodne tabele za slovenske šumnike ter uspešno implementirali ročno pretvorbo Unicode kodnih točk v UTF-8, kar je potrdilo enoličnost in zanesljivost Unicode kodiranja.


\small
\begin{thebibliography}{1}

\bibitem{ucbenik}
N. Pavešić, \emph{Informacija in kodi}, 2. spremenjena in dopolnjena izdaja.
Ljubljana: Založba Fakultete za elektrotehniko in Fakultete za računalništvo in informatiko, 2010.
ISBN 978-961-243-145-7.

\bibitem{IBM-852}
Localizely, Character Encoding: CP 852
[Na spletu]. Dostopno na: \url{https://localizely.com/character-encodings/cp852/}
[Dostopano: 22. november 2025]

\bibitem{ISO-8859-2}
\emph{ISO/IEC 8859-2:1999} Information technology — 8-bit single-byte coded graphic character sets — Part 2: Latin alphabet No. 2
[Na spletu]. Dostopno na: \url{https://www.iso.org/obp/ui/en/#iso:std:iso-iec:8859:-2:ed-1:v1:en}
[Dostopano: 22. november 2025]

\bibitem{WIN-1250}
\emph{Wikipedia} Windows-1250
[Na spletu]. Dostopno na: \url{https://en.wikipedia.org/wiki/Windows-1250}
[Dostopano: 22. november 2025]

\bibitem{MacCE}
\emph{Everything explained today} Mac OS Central European encoding explained
[Na spletu]. Dostopno na: \url{https://everything.explained.today/Mac_OS_Central_European_encoding/}
[Dostopano: 22. november 2025]

\bibitem{unicode1}
The Unicode Consortium,
\emph{The Unicode Standard, Version 15.0.0}.
Mountain View, CA: The Unicode Consortium, 2022.
ISBN 978-1-936213-32-0.

\bibitem{unicode}
The Unicode Consortium, \emph{The Unicode Standard}, v. 15.0.
[Na spletu]. Dostopno na: \url{https://www.unicode.org/standard/standard.html}

\bibitem{rfc3629}
F. Yergeau, ``UTF-8, a transformation format of ISO 10646,''
\emph{IETF RFC 3629}, nov. 2003.
[Na spletu]. Dostopno na: \url{https://www.rfc-editor.org/rfc/rfc3629}

\bibitem{python}
Python Software Foundation, \emph{Python 3 Documentation}.
[Na spletu]. Dostopno na: \url{https://docs.python.org/3/}

\end{thebibliography}

\appendix
\section{Prevedno besedilo}
\label{nal2.1}
\begin{itemize}

\item UTF-8 je eden izmed načinov kodiranja mednarodnega nabora znakov unicode, pri katerem znaki ASCII ostanejo enozložni, ostali znaki pa lahko zasedajo več zlogov.

\item {\cjkfont UTF-8(8-bit Unicode Transformation Format)\\是一種針對Unicode的可變長度字元編碼,\\也是一种前缀码。}

\item {\cjkfont UTF-8은 유니코드를 위한 가변 길이 문자 인코딩 방식 중 하나로, 켄 톰프슨과 롭 파이크가 만들었다.}

\item {\grfont Το UTF-8 (8-bit Unicode Transformation Format) είναι ένα μη-απωλεστικό σχήμα κωδικοποίησης χαρακτήρων μεταβλητού μήκους για το πρότυπο Unicode που δημιουργήθηκε από τους Ken Thompson και Rob Pike.}
\end{itemize}

\section{Tabele}
\begin{table}[H]
\centering
\begin{tabular}{|c|c|c|c|}
\hline
 znak & deset. & šest. & binarno \\ \hline
 '$\backslash n$' & 10 & 0A & 00001010 \\
 '$\backslash r$' & 13 & 0D & 00001101 \\
 ' ' & 32 & 20 & 00100000  \\
 '(' & 40 & 28 & 00101000  \\
 ')' & 41 & 29 & 00101001  \\
 ',' & 44 & 2C & 00101100  \\
 '-' & 45 & 2D & 00101101  \\
 '.' & 46 & 2E & 00101110  \\
 '8' & 56 & 38 & 00111000  \\
 'A' & 65 & 41 & 01000001  \\
 'C' & 67 & 43 & 01000011  \\
 'F' & 70 & 46 & 01000110  \\
 'I' & 73 & 49 & 01001001  \\
 'K' & 75 & 4B & 01001011  \\
 'P' & 80 & 50 & 01010000  \\
 'R' & 82 & 52 & 01010010  \\
 'S' & 83 & 53 & 01010011  \\
 'T' & 84 & 54 & 01010100  \\
 'U' & 85 & 55 & 01010101  \\
 'a' & 97 & 61 & 01100001  \\
 'b' & 98 & 62 & 01100010  \\
 'c' & 99 & 63 & 01100011  \\
 'd' & 100 & 64 & 01100100 \\
 'e' & 101 & 65 & 01100101 \\
 'f' & 102 & 66 & 01100110 \\
 'g' & 103 & 67 & 01100111 \\
 'h' & 104 & 68 & 01101000 \\
 'i' & 105 & 69 & 01101001 \\
 'j' & 106 & 6A & 01101010 \\
 'k' & 107 & 6B & 01101011 \\
 'l' & 108 & 6C & 01101100 \\
 'm' & 109 & 6D & 01101101 \\
 'n' & 110 & 6E & 01101110 \\
 'o' & 111 & 6F & 01101111 \\
 'p' & 112 & 70 & 01110000 \\
 'r' & 114 & 72 & 01110010 \\
 's' & 115 & 73 & 01110011 \\
 't' & 116 & 74 & 01110100 \\
 'u' & 117 & 75 & 01110101 \\
 'v' & 118 & 76 & 01110110 \\
 'z' & 122 & 7A & 01111010 \\
 'č' & 50317 & C4 8D & 11000100 10001101 \\
 'ž' & 50622 & C5 BE & 11000101 10111110 \\ \hline

\end{tabular}
\caption{Unikatni ASCII in slovenski znaki v prevedenem besedilu ter njihove kodne vrednosti.}
\label{tab:nal2:eu}
\end{table}


\begin{table}[H]
\centering
\begin{tabular}{|c|c|c|c|}
\hline
 znak & deset. & šest. & binarno \\ \hline

 \grfont 'Τ' & 52900 & CE A4 & 11001110 10100100 \\
 \grfont 'έ' & 52909 & CE AD & 11001110 10101101 \\
 \grfont 'ή' & 52910 & CE AE & 11001110 10101110 \\
 \grfont 'ί' & 52911 & CE AF & 11001110 10101111 \\
 \grfont 'α' & 52913 & CE B1 & 11001110 10110001 \\
 \grfont 'β' & 52914 & CE B2 & 11001110 10110010 \\
 \grfont 'γ' & 52915 & CE B3 & 11001110 10110011 \\
 \grfont 'δ' & 52916 & CE B4 & 11001110 10110100 \\
 \grfont 'ε' & 52917 & CE B5 & 11001110 10110101 \\
 \grfont 'η' & 52919 & CE B7 & 11001110 10110111 \\
 \grfont 'θ' & 52920 & CE B8 & 11001110 10111000 \\
 \grfont 'ι' & 52921 & CE B9 & 11001110 10111001 \\
 \grfont 'κ' & 52922 & CE BA & 11001110 10111010 \\
 \grfont 'λ' & 52923 & CE BB & 11001110 10111011 \\
 \grfont 'μ' & 52924 & CE BC & 11001110 10111100 \\
 \grfont 'ν' & 52925 & CE BD & 11001110 10111101 \\
 \grfont 'ο' & 52927 & CE BF & 11001110 10111111 \\
 \grfont 'π' & 53120 & CF 80 & 11001111 10000000 \\
 \grfont 'ρ' & 53121 & CF 81 & 11001111 10000001 \\
 \grfont 'ς' & 53122 & CF 82 & 11001111 10000010 \\
 \grfont 'σ' & 53123 & CF 83 & 11001111 10000011 \\
 \grfont 'τ' & 53124 & CF 84 & 11001111 10000100 \\
 \grfont 'υ' & 53125 & CF 85 & 11001111 10000101 \\
 \grfont 'χ' & 53127 & CF 87 & 11001111 10000111 \\
 \grfont 'ω' & 53129 & CF 89 & 11001111 10001001 \\
 \grfont 'ό' & 53132 & CF 8C & 11001111 10001100 \\
 \grfont 'ύ' & 53133 & CF 8D & 11001111 10001101 \\ \hline

\end{tabular}
\caption{Unikatni grški znaki v prevedenem besedilu in njihove kodne vrednosti.}
\label{tab:nal2:greek}
\end{table}


\begin{table}[H]
\centering
\begin{tabular}{|c|c|c|c|}
\hline
 znak & deset. & šest. & binarno \\ \hline

 \cjkfont '。' & 14909570 & E3 80 82 & 11100011 10000000 10000010 \\
 \cjkfont '一' & 14989440 & E4 B8 80 & 11100100 10111000 10000000 \\
 \cjkfont '也' & 14989727 & E4 B9 9F & 11100100 10111001 10011111 \\
 \cjkfont '元' & 15041923 & E5 85 83 & 11100101 10000101 10000011 \\
 \cjkfont '前' & 15042957 & E5 89 8D & 11100101 10001001 10001101 \\
 \cjkfont '可' & 15044527 & E5 8F AF & 11100101 10001111 10101111 \\
 \cjkfont '字' & 15052183 & E5 AD 97 & 11100101 10101101 10010111 \\
 \cjkfont '對' & 15052941 & E5 B0 8D & 11100101 10110000 10001101 \\
 \cjkfont '度' & 15055526 & E5 BA A6 & 11100101 10111010 10100110 \\
 \cjkfont '是' & 15112367 & E6 98 AF & 11100110 10011000 10101111 \\
 \cjkfont '的' & 15178372 & E7 9A 84 & 11100111 10011010 10000100 \\
 \cjkfont '码' & 15179905 & E7 A0 81 & 11100111 10100000 10000001 \\
 \cjkfont '碼' & 15180476 & E7 A2 BC & 11100111 10100010 10111100 \\
 \cjkfont '种' & 15181709 & E7 A7 8D & 11100111 10100111 10001101 \\
 \cjkfont '種' & 15181998 & E7 A8 AE & 11100111 10101000 10101110 \\
 \cjkfont '編' & 15185832 & E7 B7 A8 & 11100111 10110111 10101000 \\
 \cjkfont '缀' & 15187072 & E7 BC 80 & 11100111 10111100 10000000 \\
 \cjkfont '變' & 15249034 & E8 AE 8A & 11101000 10101110 10001010 \\
 \cjkfont '針' & 15304605 & E9 87 9D & 11101001 10000111 10011101 \\
 \cjkfont '長' & 15308215 & E9 95 B7 & 11101001 10010101 10110111 \\ \hline

\end{tabular}
\caption{Unikatni kitajski znaki (CJK) v prevedenem besedilu in njihove kodne vrednosti.}
\label{tab:nal2:chinese}
\end{table}

\clearpage

\begin{table}[H]
\centering
\begin{tabular}{|c|c|c|c|}
\hline
 znak & deset. & šest. & binarno \\ \hline

 \cjkfont '가' & 15380608 & EA B0 80 & 11101010 10110000 10000000 \\
 \cjkfont '과' & 15381436 & EA B3 BC & 11101010 10110011 10111100 \\
 \cjkfont '길' & 15382712 & EA B8 B8 & 11101010 10111000 10111000 \\
 \cjkfont '나' & 15434392 & EB 82 98 & 11101011 10000010 10011000 \\
 \cjkfont '니' & 15436680 & EB 8B 88 & 11101011 10001011 10001000 \\
 \cjkfont '다' & 15436708 & EB 8B A4 & 11101011 10001011 10100100 \\
 \cjkfont '드' & 15438748 & EB 93 9C & 11101011 10010011 10011100 \\
 \cjkfont '들' & 15438756 & EB 93 A4 & 11101011 10010011 10100100 \\
 \cjkfont '딩' & 15439017 & EB 94 A9 & 11101011 10010100 10101001 \\
 \cjkfont '로' & 15442332 & EB A1 9C & 11101011 10100001 10011100 \\
 \cjkfont '롭' & 15442349 & EB A1 AD & 11101011 10100001 10101101 \\
 \cjkfont '를' & 15443388 & EB A5 BC & 11101011 10100101 10111100 \\
 \cjkfont '만' & 15443852 & EB A7 8C & 11101011 10100111 10001100 \\
 \cjkfont '문' & 15445176 & EB AC B8 & 11101011 10101100 10111000 \\
 \cjkfont '방' & 15446185 & EB B0 A9 & 11101011 10110000 10101001 \\
 \cjkfont '변' & 15446912 & EB B3 80 & 11101011 10110011 10000000 \\
 \cjkfont '슨' & 15501992 & EC 8A A8 & 11101100 10001010 10101000 \\
 \cjkfont '식' & 15502237 & EC 8B 9D & 11101100 10001011 10011101 \\
 \cjkfont '었' & 15505288 & EC 97 88 & 11101100 10010111 10001000 \\
 \cjkfont '위' & 15506564 & EC 9C 84 & 11101100 10011100 10000100 \\
 \cjkfont '유' & 15506592 & EC 9C A0 & 11101100 10011100 10100000 \\
 \cjkfont '은' & 15506816 & EC 9D 80 & 11101100 10011101 10000000 \\
 \cjkfont '이' & 15506868 & EC 9D B4 & 11101100 10011101 10110100 \\
 \cjkfont '인' & 15506872 & EC 9D B8 & 11101100 10011101 10111000 \\
 \cjkfont '자' & 15507088 & EC 9E 90 & 11101100 10011110 10010000 \\
 \cjkfont '중' & 15508625 & EC A4 91 & 11101100 10100100 10010001 \\
 \cjkfont '켄' & 15514756 & EC BC 84 & 11101100 10111100 10000100 \\
 \cjkfont '코' & 15515028 & EC BD 94 & 11101100 10111101 10010100 \\
 \cjkfont '크' & 15565228 & ED 81 AC & 11101101 10000001 10101100 \\
 \cjkfont '톰' & 15566512 & ED 86 B0 & 11101101 10000110 10110000 \\
 \cjkfont '파' & 15568012 & ED 8C 8C & 11101101 10001100 10001100 \\
 \cjkfont '프' & 15570052 & ED 94 84 & 11101101 10010100 10000100 \\
 \cjkfont '하' & 15570328 & ED 95 98 & 11101101 10010101 10011000 \\
 \cjkfont '한' & 15570332 & ED 95 9C & 11101101 10010101 10011100 \\
 \cjkfont '(' & 15711368 & EF BC 88 & 11101111 10111100 10001000 \\
 \cjkfont ')' & 15711369 & EF BC 89 & 11101111 10111100 10001001 \\
 \cjkfont ',' & 15711372 & EF BC 8C & 11101111 10111100 10001100 \\ \hline

\end{tabular}
\caption{Unikatni korejski znaki (Hangul) v prevedenem besedilu in njihove kodne vrednosti.}
\label{tab:nal2:korean}
\end{table}

\end{document}
